\documentclass[]{article}
\usepackage{lmodern}
\usepackage{amssymb,amsmath}
\usepackage{ifxetex,ifluatex}
\usepackage{fixltx2e} % provides \textsubscript
\ifnum 0\ifxetex 1\fi\ifluatex 1\fi=0 % if pdftex
  \usepackage[T1]{fontenc}
  \usepackage[utf8]{inputenc}
\else % if luatex or xelatex
  \ifxetex
    \usepackage{mathspec}
  \else
    \usepackage{fontspec}
  \fi
  \defaultfontfeatures{Ligatures=TeX,Scale=MatchLowercase}
\fi
% use upquote if available, for straight quotes in verbatim environments
\IfFileExists{upquote.sty}{\usepackage{upquote}}{}
% use microtype if available
\IfFileExists{microtype.sty}{%
\usepackage{microtype}
\UseMicrotypeSet[protrusion]{basicmath} % disable protrusion for tt fonts
}{}
\usepackage[margin=1in]{geometry}
\usepackage{hyperref}
\hypersetup{unicode=true,
            pdftitle={Introduction to MATH 615},
            pdfborder={0 0 0},
            breaklinks=true}
\urlstyle{same}  % don't use monospace font for urls
\usepackage{graphicx,grffile}
\makeatletter
\def\maxwidth{\ifdim\Gin@nat@width>\linewidth\linewidth\else\Gin@nat@width\fi}
\def\maxheight{\ifdim\Gin@nat@height>\textheight\textheight\else\Gin@nat@height\fi}
\makeatother
% Scale images if necessary, so that they will not overflow the page
% margins by default, and it is still possible to overwrite the defaults
% using explicit options in \includegraphics[width, height, ...]{}
\setkeys{Gin}{width=\maxwidth,height=\maxheight,keepaspectratio}
\IfFileExists{parskip.sty}{%
\usepackage{parskip}
}{% else
\setlength{\parindent}{0pt}
\setlength{\parskip}{6pt plus 2pt minus 1pt}
}
\setlength{\emergencystretch}{3em}  % prevent overfull lines
\providecommand{\tightlist}{%
  \setlength{\itemsep}{0pt}\setlength{\parskip}{0pt}}
\setcounter{secnumdepth}{0}
% Redefines (sub)paragraphs to behave more like sections
\ifx\paragraph\undefined\else
\let\oldparagraph\paragraph
\renewcommand{\paragraph}[1]{\oldparagraph{#1}\mbox{}}
\fi
\ifx\subparagraph\undefined\else
\let\oldsubparagraph\subparagraph
\renewcommand{\subparagraph}[1]{\oldsubparagraph{#1}\mbox{}}
\fi

%%% Use protect on footnotes to avoid problems with footnotes in titles
\let\rmarkdownfootnote\footnote%
\def\footnote{\protect\rmarkdownfootnote}

%%% Change title format to be more compact
\usepackage{titling}

% Create subtitle command for use in maketitle
\newcommand{\subtitle}[1]{
  \posttitle{
    \begin{center}\large#1\end{center}
    }
}

\setlength{\droptitle}{-2em}

  \title{Introduction to MATH 615}
    \pretitle{\vspace{\droptitle}\centering\huge}
  \posttitle{\par}
    \author{}
    \preauthor{}\postauthor{}
      \predate{\centering\large\emph}
  \postdate{\par}
    \date{Last Updated 2018-08-22 12:28:54}


\begin{document}
\maketitle

\section{Navigating course resources}\label{navigating-course-resources}

\begin{itemize}
\tightlist
\item
  My website: \url{http://norcalbiostat.com}

  \begin{itemize}
  \tightlist
  \item
    First stop for all class materials
  \item
    Links to individual classes, but shared resources on programming and
    data.
  \end{itemize}
\item
  This class website \url{https://norcalbiostat.github.io/MATH615/}

  \begin{itemize}
  \tightlist
  \item
    Landing page for announcements
  \item
    Details on weekly topics can be found on the schedule
  \item
    Includes links to notes, assignments and additional materials
  \end{itemize}
\item
  Both are mobile friendly.
\item
  Often links will be broken. Typo's happen. Notify me immediately and
  I'll get to it asap.
\item
  The syllabus covers course details such as grading, office location
  and classroom policies.
\end{itemize}

\section{Online systems}\label{online-systems}

\begin{itemize}
\tightlist
\item
  Blackboard Learn (BBL) used for recording grades only.
\item
  Assignments will be turned and peer reviewed through Google Drive.
\item
  The textbook used for data, reading and learning content. Sometimes
  problem sets.

  \begin{itemize}
  \tightlist
  \item
    Great long term resource, but a new edition will be coming out next
    year.
  \end{itemize}
\item
  Slack will be used for outside class discussions, homework help and
  general chatter.

  \begin{itemize}
  \tightlist
  \item
    I will not answer programming questions through email.
  \item
    Download either the phone app or the desktop app (I use both)
  \item
    Push notifications are not automatically turned on. You must opt in.
  \end{itemize}
\end{itemize}

\section{Assignment process}\label{assignment-process}

\begin{itemize}
\tightlist
\item
  When available, use the template provided.
\item
  All assignments are uploaded as PDF's to our shared Google Drive.
\item
  Each assignment has it's own folder.
\item
  Assignments are uploaded to the \emph{Incoming} folder, and returned
  in the \emph{Reviewed} folder.
\item
  File naming convention: \texttt{userid\_assignment.pdf} i.e.
  \texttt{rdonatello\_citation.pdf}
\item
  No special characters or spaces in your file names.
\item
  Each assignment will be graded by me, and peer reviewed by 2 of your
  classmates.
\end{itemize}

\subsection{Peer review}\label{peer-review}

\begin{itemize}
\tightlist
\item
  The peer rotation schedule can be found in our
  \href{https://drive.google.com/open?id=0B2-FPH8JVeuebFpnV3RZaU9KbTg}{Google
  Drive}.
\item
  After the assignment due date, go to the \emph{Incoming} folder and
  \textbf{download} a copy of each of the 2 assignments you are
  scheduled to review onto your computer.

  \begin{itemize}
  \tightlist
  \item
    Do not edit directly in Google Docs unless you have already made a
    copy
  \item
    File naming convention:
    \texttt{rdonatello\_citation\_rv\_ricardo.pdf}
  \end{itemize}
\item
  Use the commenting features of Adobe PDF to make any corrections or
  comments

  \begin{itemize}
  \tightlist
  \item
    You must mention \textbf{two} things that are good about the
    assignment and \textbf{two} things that need improvement or
    corrections.
  \item
    Be specific about each edit. Don't just say ``great job''.
  \item
    Give the type of feedback and review that you want others to give to
    you.
  \end{itemize}
\end{itemize}

\section{Descriptive vs.~Inferential
Statistics}\label{descriptive-vs.inferential-statistics}

\begin{itemize}
\tightlist
\item
  Two main phases of Statistics.

  \begin{itemize}
  \tightlist
  \item
    Also called Exploratory and Confirmatory
  \end{itemize}
\item
  \href{../reading/PDS_Intro_Stat.pdf}{Passion Driven Statistics}

  \begin{itemize}
  \tightlist
  \item
    Backbone theory behind the class. Read this PDF to get a sense of
    how statistics fits into science.
  \item
    The videos can be used as an additional learning tool.
  \end{itemize}
\item
  Google \texttt{data\ analysis\ lifecycle} and look at
  \href{https://www.google.com/search?q=data+analytics+life+cycle\&newwindow=1\&source=lnms\&tbm=isch\&sa=X\&ved=0ahUKEwi32cepp9DcAhXLhVQKHYR8AScQ_AUICigB\&biw=1482\&bih=876}{images}.
  What sense do you get?
\end{itemize}

\section{Computing and
Reproducibility}\label{computing-and-reproducibility}

\begin{itemize}
\tightlist
\item
  No more TI-83, modern statistics is computational based.
\item
  Big push for open research in the Natural Sciences.

  \begin{itemize}
  \tightlist
  \item
    Sharing code \& data. Sometimes required along with manuscript for
    publishing.
  \end{itemize}
\item
  Reproducibility. Give someone else access to your data and code, and
  they can replicate your findings.

  \begin{itemize}
  \tightlist
  \item
    We will practice this in this class.
  \item
    I practice this by putting all class material online with a cc-by
    license. (others are free to copy and share my work with
    acknowledgement)
  \end{itemize}
\item
  Review these
  \href{http://benmarwick.github.io/UW-eScience-reproducibility-social-sciences/\#/}{Slides}
  on reproducible research in the social sciences.

  \begin{itemize}
  \tightlist
  \item
    I will not require any measure of version control or open source
    coding in this class.
  \end{itemize}
\item
  Be mindful about file naming conventions (slide 11). Make a plan and
  stick with it.

  \begin{itemize}
  \tightlist
  \item
    \url{https://www.xkcd.com/1459/}
  \end{itemize}
\item
  Expect to bring your laptop every day to class.

  \begin{itemize}
  \tightlist
  \item
    The more reading and content learning done outside of class, the
    more time for in class analysis and discussion
  \end{itemize}
\end{itemize}

\section{Software program of choice
(SPC)}\label{software-program-of-choice-spc}

\begin{itemize}
\tightlist
\item
  This class is not a class on how to use the software program. You will
  be learning that on your own or in another class.
\item
  All my lecture notes use R. This entire website is built with R. R is
  a pioneer in generating reproducible and publishable quality reports.

  \begin{itemize}
  \tightlist
  \item
    \href{../reading/Final_chem_report.pdf}{Here's an student-generated
    example}
  \end{itemize}
\item
  I will not dictate which software program you use in this class.
\item
  But I will expect you to navigate and use code. You can point and
  click your way to an answer, but code must be saved and reusible with
  minimal changes.
\item
  Be open to new things, there is power in being polyglottal.
\item
  The first few weeks will be ramp up time.
\end{itemize}

\subsection{SPSS}\label{spss}

\begin{itemize}
\tightlist
\item
  Purchase v24 or v25 from
  \url{http://www-03.ibm.com/software/products/en/spss-stats-gradpack}
  for \$50 for 6mo rental.
\item
  Point and click, but can save code and write scripts.
\item
  Stand alone program. No integration. Licenses are not cheap.
\item
  Will be used again in NSFC 600 (no exp necc for that class either)
\item
  On campus resources: From the desk of David Philhour (BSS)

  \begin{itemize}
  \tightlist
  \item
    Open computer labs in Butte Hall (207, 211) with many open lab
    hours.
  \item
    Tutoring center in AJH108 run by Dr.~Penelope Kuhn.
  \item
    The psyc depth lab is Modoc 224 but is pretty impacted with classes
    Monday thru Thursday. Friday's are pretty open.
  \end{itemize}
\item
  Off Campus resources

  \begin{itemize}
  \tightlist
  \item
    Kent State University Tutorials:
    \url{https://libguides.library.kent.edu/SPSS/home}
  \item
    UCLA Institute for Digital Research and Education:
    \url{https://stats.idre.ucla.edu/spss/}
  \item
    Recommended selection of YouTube videos
    \url{https://www.youtube.com/results?search_query=andy+field+spss+tutorials}
  \end{itemize}
\end{itemize}

\subsection{R}\label{r}

\begin{itemize}
\tightlist
\item
  Free. Installation Instructions here:
  \url{https://norcalbiostat.netlify.com/post/software-overview/}
\item
  Harder up front, more powerful in the end.
\item
  Seamless integration with a multitude of other scientific analysis and
  reproducible reporting mechanisms.
\item
  Becoming much more popular in all scientific fields of study. One of
  the primary languages for Data Science.
\item
  Google at \href{https://osf.io/69gub/wiki/home/}{diagram} of the
  \texttt{tidyverse} (a suite of functions in R). Compare it to the
  images of the data analysis life cycle. What sense do you get?
\item
  Need some motivation?

  \begin{itemize}
  \tightlist
  \item
    \url{https://www.psychologicalscience.org/observer/why-you-should-become-a-user-a-brief-introduction-to-r}
  \item
    \url{https://osf.io/j28w7/}
  \item
    \url{https://www.youtube.com/watch?v=jn_3N_o2d6Q}
  \end{itemize}
\item
  On campus resources

  \begin{itemize}
  \tightlist
  \item
    Introduction to R (MATH 130) 1 unit CR/NC
  \item
    Data Science Initiative workshops, talks, open drop in analysis
    time.
  \end{itemize}
\item
  Off Campus resources (a few)

  \begin{itemize}
  \tightlist
  \item
    Chico R Users Group

    \begin{itemize}
    \tightlist
    \item
      \href{https://www.meetup.com/Chico-R-Users-Group/}{Meetup}
    \item
      \href{https://groups.google.com/forum/\#!forum/chico-rug}{Google
      l-serv}
    \end{itemize}
  \item
    \href{http://www.datacamp.com}{Data camp}
  \item
    \href{http://www.statmethods.net/}{Quick-R}
  \item
    \href{http://www.cookbook-r.com/}{Cookbook for R}
  \item
    \href{http://dwoll.de/rexrepos/}{R Examples Repository} (This site
    was also built using R Markdown, is open source and a fabulous
    example of reproducible research!)
  \end{itemize}
\end{itemize}

\subsection{SAS? STATA? Python?}\label{sas-stata-python}

\begin{itemize}
\tightlist
\item
  SAS has only now working on literate and integrated programming by
  using Jupyter notebooks and SAS University Edition (free)
\item
  Stata has a few user written packages that allow for the integration
  of LaTeX or markdown into your code document.
\item
  Python is the other primary language for Data Science. I am in the
  process of learning Python but the capabilities are very great. If
  you're thinking this route be sure to use Jupyter notebooks.
\end{itemize}


\end{document}
