\documentclass[]{article}
\usepackage{lmodern}
\usepackage{amssymb,amsmath}
\usepackage{ifxetex,ifluatex}
\usepackage{fixltx2e} % provides \textsubscript
\ifnum 0\ifxetex 1\fi\ifluatex 1\fi=0 % if pdftex
  \usepackage[T1]{fontenc}
  \usepackage[utf8]{inputenc}
\else % if luatex or xelatex
  \ifxetex
    \usepackage{mathspec}
  \else
    \usepackage{fontspec}
  \fi
  \defaultfontfeatures{Ligatures=TeX,Scale=MatchLowercase}
\fi
% use upquote if available, for straight quotes in verbatim environments
\IfFileExists{upquote.sty}{\usepackage{upquote}}{}
% use microtype if available
\IfFileExists{microtype.sty}{%
\usepackage{microtype}
\UseMicrotypeSet[protrusion]{basicmath} % disable protrusion for tt fonts
}{}
\usepackage[margin=1in]{geometry}
\usepackage{hyperref}
\hypersetup{unicode=true,
            pdftitle={Preparation questions for Data Management},
            pdfborder={0 0 0},
            breaklinks=true}
\urlstyle{same}  % don't use monospace font for urls
\usepackage{graphicx,grffile}
\makeatletter
\def\maxwidth{\ifdim\Gin@nat@width>\linewidth\linewidth\else\Gin@nat@width\fi}
\def\maxheight{\ifdim\Gin@nat@height>\textheight\textheight\else\Gin@nat@height\fi}
\makeatother
% Scale images if necessary, so that they will not overflow the page
% margins by default, and it is still possible to overwrite the defaults
% using explicit options in \includegraphics[width, height, ...]{}
\setkeys{Gin}{width=\maxwidth,height=\maxheight,keepaspectratio}
\IfFileExists{parskip.sty}{%
\usepackage{parskip}
}{% else
\setlength{\parindent}{0pt}
\setlength{\parskip}{6pt plus 2pt minus 1pt}
}
\setlength{\emergencystretch}{3em}  % prevent overfull lines
\providecommand{\tightlist}{%
  \setlength{\itemsep}{0pt}\setlength{\parskip}{0pt}}
\setcounter{secnumdepth}{0}
% Redefines (sub)paragraphs to behave more like sections
\ifx\paragraph\undefined\else
\let\oldparagraph\paragraph
\renewcommand{\paragraph}[1]{\oldparagraph{#1}\mbox{}}
\fi
\ifx\subparagraph\undefined\else
\let\oldsubparagraph\subparagraph
\renewcommand{\subparagraph}[1]{\oldsubparagraph{#1}\mbox{}}
\fi

%%% Use protect on footnotes to avoid problems with footnotes in titles
\let\rmarkdownfootnote\footnote%
\def\footnote{\protect\rmarkdownfootnote}

%%% Change title format to be more compact
\usepackage{titling}

% Create subtitle command for use in maketitle
\newcommand{\subtitle}[1]{
  \posttitle{
    \begin{center}\large#1\end{center}
    }
}

\setlength{\droptitle}{-2em}

  \title{Preparation questions for Data Management}
    \pretitle{\vspace{\droptitle}\centering\huge}
  \posttitle{\par}
    \author{}
    \preauthor{}\postauthor{}
    \date{}
    \predate{}\postdate{}
  

\begin{document}
\maketitle

You know what variables you want to use, and you've looked over the
codebook enough now that you have an idea of some potential data issues
that you will have to address.

Here are some questions for you to think about in preparation for doing
data management on your own set of variables.

Answer the following questions in prepartion to learn how to tackle
problems such as missing data and recoding variables in the next week.
Some of these items guide you to updating your personal codebook. For
each variable that you have to modify, write a note in your codebook
what you plan to do and why. You will include these notes into your data
management script file as a record of your changes.

\begin{enumerate}
\def\labelenumi{\arabic{enumi}.}
\tightlist
\item
  \textbf{Do you understand what each variable is actually measuring?}
\end{enumerate}

\begin{itemize}
\tightlist
\item
  In your codebook, give labels to All your variables. This includes
  adding in a note for each variable identifying if it is Quantitative
  or Categorical, and Level of Measurement (i.e., Nominal, Ordinal,
  Interval, and Ratio).
\item
  Does your SPC recognize these variables as such: If not, note down
  that you'll have to change this later.
\end{itemize}

\begin{enumerate}
\def\labelenumi{\arabic{enumi}.}
\setcounter{enumi}{1}
\tightlist
\item
  \textbf{Do you need to code out missing data?}
\end{enumerate}

\begin{itemize}
\tightlist
\item
  Go through your codebook for each variable and treat ALL variables for
  missing data. If there are no missing values then write that.
\end{itemize}

\begin{enumerate}
\def\labelenumi{\arabic{enumi}.}
\setcounter{enumi}{2}
\tightlist
\item
  \textbf{Are you going to only look at a subset of individuals?}
\end{enumerate}

\begin{itemize}
\tightlist
\item
  What value of which variable specifically are you going to filter on?
\end{itemize}

\begin{enumerate}
\def\labelenumi{\arabic{enumi}.}
\setcounter{enumi}{3}
\item
  \textbf{Do you need to code out skip patterns?} Are you looking at
  questions that only pertain to a specific subpopulation? (i.e.~number
  of packs smoked per week only applies to smokers). Responses to the \#
  of packs question should be set to missing for non-smokers.
\item
  \textbf{Do you need to make response codes more logical?} Some
  examples include:
\end{enumerate}

\begin{itemize}
\tightlist
\item
  Think about how the ``yes''" and ``no''" variables are coded

  \begin{itemize}
  \tightlist
  \item
    Does NO = 0 and YES = 1?\\
  \end{itemize}
\item
  Think about how the ``strongly agree'' to ``strongly disagree''
  variables are coded

  \begin{itemize}
  \tightlist
  \item
    Do the numbers make sense?\\
  \end{itemize}
\item
  Consider recoding a quantitative variable into a categorical
  variable\\
\item
  Consider collapsing across categories

  \begin{itemize}
  \tightlist
  \item
    maybe going from 5 categories for strongly agree, agree, neutral,
    disagree, strongly disagree to 3 categories that represent strongly
    agree and agree as one category, disagree and strongly disagree as
    another, then neutral still in the middle.\\
  \end{itemize}
\item
  Consider collapsing a quantitative variable into categories based on
  percentages of the data you find after examining the frequency table.

  \begin{itemize}
  \tightlist
  \item
    (i.e.~BMI: \textless{}16: underweight, 16-18.5 normal, 18.5-25
    overweight, 30+ obese)
  \end{itemize}
\end{itemize}

\begin{enumerate}
\def\labelenumi{\arabic{enumi}.}
\setcounter{enumi}{5}
\tightlist
\item
  \textbf{Do you need to create secondary variables?} If necessary,
  create secondary variables from continuous variables. If you are
  working with a number of items that represent a single construct, it
  may be useful to create a composite variable/score.

  \begin{itemize}
  \tightlist
  \item
    For example, I want to use a list of nicotine dependence symptoms
    meant to address the presence or absence of nicotine dependence
    (i.e., tolerance, withdrawal, craving, etc.). Rather than using a
    dichotomous variable (i.e., nicotine dependence present/absent), I
    want to examine the construct as a dimensional scale (i.e., number
    of nicotine dependence symptoms). In this case, I would want to
    recode each symptom variable so that YES = 1 and NO = 0 and then sum
    the items so that they represent one composite score. In the code
    below, the nd\_sum is the new variable I am creating and the
    variables after of are the variables I am totaling up.
  \item
    nd\_sum=sum (of nd\_symptom1 nd\_symptom2 nd\_symptom3
    nd\_symptom4); (\emph{SAS})
  \end{itemize}
\end{enumerate}

\begin{itemize}
\tightlist
\item
  Don't forget when creating secondary variables, you need to visually
  check your new variables to ensure what you thought you did actually
  did what you wanted to.
\end{itemize}


\end{document}
